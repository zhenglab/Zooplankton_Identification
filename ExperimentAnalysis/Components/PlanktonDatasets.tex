\section{浮游生物数据集}
\url{https://www.nodc.noaa.gov/General/plankton.html}

伍兹霍尔海洋研究所(Woods Hole Oceanographic Institution)是专注于海洋科学与海洋工程的非盈利私人研究和教学机构。plankton:\url{https://darchive.mblwhoilibrary.org/handle/1912/7341}。

\subsection{浮游植物}

1. 论文Automated taxonomic classification of phytoplankton sampled with imaging-in-flow cytometry中用到的数据集:\url{http://aslo.org/lomethods/free/2007/0204a1.html}(采用的是Imaging FlowCytobot(IFCB))。2004和2005春天在Woods Hole Harbor采集的。

2. Martha’s Vineyard Coastal Observatory (MVCO)(采用的是Imaging FlowCytobot(IFCB)):\url{http://ifcb-data.whoi.edu/mvco}。

3. NASA Healy Arctic cruise :\url{http://ifcb-data.whoi.edu/Healy1101}

4. Salt Pond:\url{http://ifcb-data.whoi.edu/saltpond}。

5. WHOI:\url{https://darchive.mblwhoilibrary.org/handle/1912/7341}


\subsection{浮游动物}

\subsubsection{已经分类好}
1. Scientific Committee on Oceanic Research(SCOR)是由International Council for Science(ICSU)组织的来处理各个学科之间的海洋科学问题。Scientific Committee on Oceanic Research (SCOR) created an international working group to evaluate the state of Automatic Visual Plankton Identification (\url{http://www.scor-wg130.net})\cite{gorsky2010digital}.

数据集在SCOR的Archive-ImageDataSet中\url{http://www.scor-wg130.net/index.cfm?err=&CFID=21726107&CFTOKEN=fdcee774fe9206e5-C4B4A0EB-155D-0102-8481F9D3D8D047CF}

2. kaggle plankton: \url{https://www.kaggle.com/c/datasciencebowl}

\subsubsection{没有进行分类}

\subsubsection{没有找到数据集}
{\color{red}1. Zooplankton of the San Diego Region 圣迭戈(美国加利福尼亚州的一个太平洋沿岸城市)的浮游动物:\url{https://scripps.ucsd.edu/zooplanktonguide/}。

2. Coastal \& Oceanic Plankton Ecology, Production \& Observation Database (COPEPOD):\url{http://www.st.nmfs.noaa.gov/copepod/}。

3. Plankton Web:\url{http://www.sfrc.ufl.edu/planktonweb/index.htm}。
}



















