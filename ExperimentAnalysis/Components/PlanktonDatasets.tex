\section{浮游生物数据集}
\url{https://www.nodc.noaa.gov/General/plankton.html}

Imaging FlowCytobot (IFCB)

Video Plankton Recorder (VPR)最初由WHOI设计。


\subsection{浮游植物}

\subsubsection{已经分类}
1. 论文Automated taxonomic classification of phytoplankton sampled with imaging-in-flow cytometry中用到的数据集:\url{http://aslo.org/lomethods/free/2007/0204a1.html}(采用的是Imaging FlowCytobot(IFCB))。2004和2005春天在Woods Hole Harbor采集的。(已经下载)

2. 伍兹霍尔海洋研究所(Woods Hole Oceanographic Institution, WHOI)采用IFCB收集的Martha's Vineyard Coastal Observatory (MVCO)从2006到2014年的图像数据:\url{https://darchive.mblwhoilibrary.org/handle/1912/7341}。(已经下载)

\subsubsection{没有进行分类}
1. Martha’s Vineyard Coastal Observatory (MVCO)(采用的是Imaging FlowCytobot(IFCB)):\url{http://ifcb-data.whoi.edu/mvco}。(未下载)



\subsubsection{没有找到数据集}
1. Scripps Plankton Camera (SPC):\url{http://spc.ucsd.edu/imagedata/spcview-plankton-camera-image-browser/}。

2. JODC Plankton Dataset:\url{http://ecologicaldata.org/wiki/jodc-plankton-dataset}



1. NASA Healy Arctic cruise :\url{http://ifcb-data.whoi.edu/Healy1101}

2. Salt Pond:\url{http://ifcb-data.whoi.edu/saltpond}。




\subsection{浮游动物}
In the past three decades various optical technologies capable of imaging zooplankton have been developed, including bench-top type imaging systems such as ZooScan and FlowCAM as well as in situ systems such as the Video Plankton Recorder (VPR) , Underwater Vision Profiler (UVP), ZOOplankton VISualization system (ZOOVIS), the Lightframe On-sight Keyspecies Investigate System (LOKI), Shadow Image Particle Profiling Evaluation Recorder (SIPPER), and the In Situ Ichthyoplankton Imaging System (ISIIS).

\subsubsection{已经分类好}
1. Scientific Committee on Oceanic Research(SCOR)是由International Council for Science(ICSU)组织的来处理各个学科之间的海洋科学问题。Scientific Committee on Oceanic Research (SCOR) created an international working group to evaluate the state of Automatic Visual Plankton Identification (\url{http://www.scor-wg130.net})\cite{gorsky2010digital}。

数据集在SCOR的Archive-ImageDataSet中\url{http://www.scor-wg130.net/index.cfm?err=&CFID=21726107&CFTOKEN=fdcee774fe9206e5-C4B4A0EB-155D-0102-8481F9D3D8D047CF}。(已经下载)

2. ZOOSCAN: \url{http://www.zooscan.obs-vlfr.fr//} Training Sets中有training set、test set和learning set等等。(已经下载)


3. kaggle plankton: \url{https://www.kaggle.com/c/datasciencebowl},训练集已经分类,测试集没有分类。(已经下载)

4. ZooImage: \url{http://www.sciviews.org/zooimage/index.html}(需要下载)

\subsubsection{没有进行分类}

\subsubsection{没有找到数据集}
1. \url{https://www.nodc.noaa.gov/General/plankton.html}. Customized plankton datasets can also be obtained by contacting NODC User Services. \url{https://www.nodc.noaa.gov/about/contact.html}(是不是需要从这里联系去要数据)

2.The data described in this paper will shortly be made available through the CalCOFI DataZoo Website: \url{http://oceaninformatics.ucsd.edu/datazoo/}


\subsubsection{似乎是只是介绍不同种类的浮游生物}
1. MARINEBIO(似乎只是简单地介绍): \url{http://marinebio.org/oceans/zooplankton/}

2. Coastal \& Oceanic Plankton Ecology, Production \& Observation Database (COPEPOD) is an online database of plankton abundance, biomass, and composition data compiled from a global assortment of cruises, projects, and institutional holdings.(好像没有图像数据)

3. SCRIPPS INSTITUTION OF OCEANOGRAPHY——Zooplankton of the San Diego Region: \url{https://scripps.ucsd.edu/zooplanktonguide/}。浮游动物的种类分的很细,分别进行了介绍,有的种类下面都有一段小视频,可以从视频里截图(但是视频中的目标个数也不太多)。

{\color{red}1. Zooplankton of the San Diego Region 圣迭戈(美国加利福尼亚州的一个太平洋沿岸城市)的浮游动物:\url{https://scripps.ucsd.edu/zooplanktonguide/}。

2. Coastal \& Oceanic Plankton Ecology, Production \& Observation Database (COPEPOD):\url{http://www.st.nmfs.noaa.gov/copepod/}。

3. Plankton Web:\url{http://www.sfrc.ufl.edu/planktonweb/index.htm}。
}



















