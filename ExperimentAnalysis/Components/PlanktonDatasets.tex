\section{浮游生物数据集}




In the past three decades various optical technologies capable of imaging zooplankton have been developed, including bench-top type imaging systems such as ZooScan and FlowCAM as well as in situ systems such as the Video Plankton Recorder (VPR) , Underwater Vision Profiler (UVP), ZOOplankton VISualization system (ZOOVIS), the Lightframe On-sight Keyspecies Investigate System (LOKI), Shadow Image Particle Profiling Evaluation Recorder (SIPPER), and the In Situ Ichthyoplankton Imaging System (ISIIS)\cite{bi2014semi} and Imaging FlowCytobot (IFCB)


\subsection{浮游生物}

\subsubsection{已经分类}
\begin{enumerate}
\item 论文Automated taxonomic classification of phytoplankton sampled with imaging-in-flow cytometry中用到的数据集:\url{http://aslo.org/lomethods/free/2007/0204a1.html}(IFCB). The images were collected with IFCB from Woods Hole Harbor water. (已经下载)

\item 伍兹霍尔海洋研究所(Woods Hole Oceanographic Institution, WHOI): \url{https://darchive.mblwhoilibrary.org/handle/1912/7341}。The images described here are part of a much larger data set collected by IFCB at the Martha's Vineyard Coastal Observatory (MVCO) starting in 2006 and continuing to the present.(这应该就是MVCO:\url{http://ifcb-data.whoi.edu/mvco}中的数据,这里已经将2006到2014年采集的图像分类整理好,没有2015年的。图片多,正在下载)

\item plankton net: \url{http://planktonnet.awi.de}. Taxonomic database of images of plankton species.(种类已经分好了,还没找到打包下载的地方,需要抓图)。

\item Michael R. Martin's Phytoplankton Image Library: \url{http://www.cedareden.com/phyto.html}(图片不多,分类也不太准确,需要抓图)\newline

\item Scientific Committee on Oceanic Research(SCOR) created an international working group to evaluate the state of Automatic Visual Plankton Identification (\url{http://www.scor-wg130.net})\cite{gorsky2010digital}。数据集在Archive-ImageDataSet中。(已经下载)

\item ZOOSCAN: \url{http://www.zooscan.obs-vlfr.fr//} Training Sets中有几个数据集,除了之前我们实验使用的数据集,论文中的数据集(就是我们之前想和作者要的数据集)也在其中。(已经下载)

\item kaggle plankton: \url{https://www.kaggle.com/c/datasciencebowl},训练集已经分类,测试集没有分类。(已经下载)

\item ZooImage: \url{http://www.sciviews.org/zooimage/index.html}(已经下载)
\end{enumerate}

\subsubsection{没有分类}
\begin{enumerate}
\item MVCO:\url{http://ifcb-data.whoi.edu/mvco}。(2015年,需要抓图)

NASA Healy Arctic cruise :\url{http://ifcb-data.whoi.edu/Healy1101}

Salt Pond:\url{http://ifcb-data.whoi.edu/saltpond}。

\item SPC: \url{http://spc.ucsd.edu/imagedata/spcview-plankton-camera-image-browser/}。

\item \url{http://gallery.obs-vlfr.fr/gallery2/main.php}(有少量图像,少部分已经分类,大部分没有分类)
\end{enumerate}

\subsubsection{联系要数据集}
\begin{enumerate}
\item SAHFOS (Sir Alister Hardy Foundation for Ocean Science)\url{http://www.sahfos.ac.uk/pil/plankton_image_database_homepage.htm}(每一类的图片只有几张,不知道联系能不能要到更多图片)

\item \url{http://cfb.unh.edu/cfbkey/html/}(只有例图,应该有数据集)

\item plankton imaging ISIIS: \url{http://www.planktonimaging.com}(仪器)

\item Zooniverse: \url{https://www.zooniverse.org},它有一个project是Plankton Portal:\url{http://www.planktonportal.org}(仪器)

\item FlowCam: \url{http://www.fluidimaging.com/applications/aquatic-research/marine-science}(仪器)

%9. Atlantic Meridional Transect (AMT): \url{http://www.amt-uk.org/About-AMT}(应该有数据集,可以要一下)
\end{enumerate}

\subsubsection{没有找到数据集}
\begin{enumerate}
\item \url{http://life.bio.sunysb.edu/marinebio/plankton.html}(有例图,但是没有找到数据集)

\item Image Quest Marine: \url{http://www.imagequest3d.com/pictures/phytoplankton/}、\url{http://www.imagequestmarine.com/en/set/show_content_page.html?category=6&set=8&qw=}、\url{http://www.imagequest3d.com/photos/zooplankton/}(只有少量图像)

\item \url{http://australianmuseum.net.au/zooplankton}(只有例图)

\item Census of Marine Zooplankton: \url{http://www.cmarz.org/galleries.html#}(只有少量图像)

\item \url{http://habsos.noaa.gov}(网页左下角有图片,但是没有找到数据集)

\item The data described in this paper will shortly be made available through the CalCOFI DataZoo Website: \url{http://oceaninformatics.ucsd.edu/datazoo/}(这是OCEANS那篇论文提到的数据集,我还没有找到怎样下载)

\item Plankton Web:\url{http://www.sfrc.ufl.edu/planktonweb/index.htm}。(只有少量图例)

\item NOAA: \url{https://www.nodc.noaa.gov/access/cdrom.html#woa09}(DVD)(从\url{https://www.nodc.noaa.gov/General/plankton.html}找到的)
\end{enumerate}





\subsubsection{介绍浮游生物}
\begin{enumerate}
\item MARINEBIO: \url{http://marinebio.org/oceans/zooplankton/}

\item SCRIPPS INSTITUTION OF OCEANOGRAPHY——Zooplankton of the San Diego Region: \url{https://scripps.ucsd.edu/zooplanktonguide/}。浮游动物的种类分的很细,分别进行了介绍,有的种类下面都有一段小视频。

\item Zooplankton of the San Diego Region 圣迭戈(美国加利福尼亚州的一个太平洋沿岸城市)的浮游动物:\url{https://scripps.ucsd.edu/zooplanktonguide/}。

\item \url{http://www.imas.utas.edu.au/zooplankton/home}
\end{enumerate}



%2. Coastal \& Oceanic Plankton Ecology, Production \& Observation Database (COPEPOD): \url{http://www.st.nmfs.noaa.gov/copepod/}, is an online database of plankton abundance, biomass, and composition data compiled from a global assortment of cruises, projects, and institutional holdings.

%2. JODC Plankton Dataset:\url{http://ecologicaldata.org/wiki/jodc-plankton-dataset}
%7. \url{http://cs231n.stanford.edu/reports/sagarc14_final_report.pdf}和\url{http://hmsc.oregonstate.edu/research-labs/planktonlab/outreach}、\url{http://hmsc.oregonstate.edu/image-album/plankton-imagery}(没有找到数据集,可以去要)















