\section{特征(PkID)}

PkID中用到的特征一共有67个:Area, Mean, StdDev, Mode, Min, Max, X, Y, XM, YM, Perim., BX, BY, Width, Height, Major, Minor, Angle, Circ., Feret, IntDen, Median, Skew, Kurt, \%Area, XStart, YStart, Area\_exc, Fractal, Skelarea, Slope, Histcum1, Histcum2, Histcum3, XMg5, YMg5, Compentropy, Compmean, Compslope, CompM1, CompM2, CompM3, Symetrieh, Symetriev, Symetriehc, Symetrievc, Tag, ESD, Elongation, Range, MeanPos, CentroidsD, CV, SR, PerimAreaexc, FeretAreaexc, PerimFeret, PerimMaj, Circexc, CDexc, {\color{blue}Nb1, Nb2, Nb3, Convperim, Convarea, Fcons, ThickR}

从训练集的PID文件文件中看到,Compentropy, Compmean, Compslope, CompM1, CompM2, CompM3这6个特征在所有图像上的值都为0,在训练分类器时是不起作用的。这6个特征的具体含义也没有找到。

\subsection{位置特征}

\begin{description}
\item[BX] 能够包围物体,且平行于图像两条边的最小外界矩形的左上角顶点的X坐标 
\item[BY] 能够包围物体,且平行于图像两条边的最小外界矩形的左上角顶点的Y坐标 
\item[Height] 能够包围物体,且平行于图像两条边的最小外界矩形的高
\item[Width] 能够包围物体,且平行于图像两条边的最小外界矩形的宽
\item[XStart] 图像最左上角像素点的X坐标
\item[YStart] 图像最左上角像素点的Y坐标
\item[XM] 物体灰度重心的X坐标
\item[YM] 物体灰度重心的Y坐标
\item[XMg5] gamma值为51时的物体灰度重心的X坐标
\item[YMg5] gamma值为51时的物体灰度重心的Y坐标
\item[X] 物体重心点的X坐标
\item[Y] 物体重心点的Y坐标
\end{description}

\subsection{尺寸特征}

\begin{description}
\item[Area] 包含物体的最小矩形面积 
\item[Perim] 物体最外层边缘的长度
\item[Major] 物体内切椭圆的长轴
\item[Minor] 物体内切椭圆的短轴
\item[Feret] Maximum feret diameter(最大费雷特径), 沿物体边缘任意两个点的最长距离
\item[Area\_exc] 去掉物体空洞后的表面积,空洞是指灰度值与背景相同的部分
\item[\%area] 物体表面积中空洞所占的百分比,即背景所占的比例
\end{description}

\subsection{灰度值特征}

\begin{description}
\item[Min] 物体内部所有像素点的最小灰度值 (0 = black)
\item[Max] 物体内部所有像素点的最大灰度值 (255 = white)
\item[Mean] 物体内的平均灰度值; 物体中所有像素点的灰度值的总和除以总的像素个数
\item[IntDen] Integrated density(总密度)。物体内像素点的灰度值的总和($IntDen = Area * Mean$)
\item[StdDev] 物体内像素的灰度值的标准差
\item[Mode] Modal grey value within the object
\item[Skew] 灰度直方图的偏斜度,是灰度分布是否符合正态分布的检验参数
\item[Kurt] 灰度直方图的峰值 
\item[Mean\_exc] 物体内部去掉空洞后的平均灰度值 ($Mean\_exc = IntDen / Area\_exc$)
\item[Median] 物体内像素的灰度值的中位数
\item[Slope] 归一化的灰度累计直方图的斜率
\item[Histcum1] 灰度累计直方图的值为25\%时所对应的灰度值
\item[Histcum2] 灰度累计直方图的值为50\%时所对应的灰度值
\item[Histcum3] 灰度累计直方图的值为75\%时所对应的灰度值
\end{description}

\subsection{形状特征}

\begin{description}
\item[Fractal] 物体边界的分形维数 (Berube and Jebrak, 1999),表明物体边界的不规则程度
\item[Skelarea] 骨架像素的表面积(在二值图像中,不断地从物体边缘处减去像素点直到仅剩一个像素的宽度,最后所得图形的像素点数)
\item[Circ] $Circularity = (4 * Pi * Area) / Perim^2$; 表征物体接近圆的程度,值等于1时,说明物体为正圆形,值越接近0,物体体形越长。
\item[Angle] 浮游动物主轴与图片x轴形成的夹角,在图片切割后旋转图片测量相关参数使用
\item[Symetrieh] 水平对称
\item[Symetriev] 垂直对称
\item[Symetriehc]
\item[Symetrievc]
\end{description}

\subsection{自定义特征}

\begin{description}
    \item[ESD] $2 \times \sqrt{\cfrac{Area}{\pi}}$
    \item[Elongation] $\cfrac{Major}{Minor}$
    \item[Range] $Max-Min$
    \item[MeanPos] $\cfrac{Mean-Max}{Max-Min}$
    \item[CentrodisD] $\sqrt{(XM-X)^{2}+(YM-Y)^{2}}$
    \item[CV] $100 \times \cfrac{StdDev}{Mean}$
    \item[SR] $100 \times \cfrac{StdDev}{Max-Min}$
    \item[PerimAreaexc] $\cfrac{Perim}{\sqrt{Area\_exc}}$
    \item[FeretAreaexc] $\cfrac{Feret}{\sqrt{Area\_exc}}$
    \item[PerimFeret] $\cfrac{Perim}{Feret}$
    \item[PerimMaj] $\cfrac{Perim}{Major}$
    \item[Circexc] $\cfrac{4 \times \pi Area\_exc}{Perim^{2}}$
    \item[CDexc] $\cfrac{\sqrt{(XM-X)^{2}+(YM-Y)^{2}}}{{\sqrt{Area\_exc}}}$
\end{description}
