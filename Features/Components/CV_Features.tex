\section{计算机视觉特征提取}

\subsection{几何参数}

\subsubsection{边界的周长}
轮廓边界的周长。对轮廓边缘上的像素点的统计。

\subsubsection{边界的曲率}

\subsubsection{面积}
描述区域大小的特征。对区域内总像素点的统计。

\subsubsection{宽度和高度}
最小外接矩形的宽度和高度

\subsubsection{矩形度}
反映被检测目标的最小外接矩形的充满程度,当目标的形状越接近矩形时,矩形度的值越接近1。
    \begin{displaymath}
    R=\frac{A}{WH}
    \end{displaymath}
    A为目标的面积,W、H分别为最小外接矩形的宽度和高度。

\subsubsection{体态比}
为目标最小外接矩形的长与宽的比值。
    \begin{displaymath}
    C=\frac{W}{H}
    \end{displaymath}
    
\subsubsection{圆形性}
用目标区域的所有边界点定义的特征向量。
    \begin{displaymath}
    C_{I}=\frac{\mu_{R}}{\sigma_{R}}
    \end{displaymath}
    $\mu_{R}$为区域重心到边界点的平均距离,$\sigma_{R}$为从区域重心到边界点的距离的平均方差。

\subsubsection{偏心率}
在一定程度上反映了区域的紧凑程度。定义为目标区域长短主轴的平方根的比值。
    \begin{displaymath}
    E=\frac{p}{q}
    \end{displaymath}
    设目标区域在XY平面上,区域像素点绕X轴的转动惯量为A,绕Y轴的转动惯量为B,惯性积为C。目标区域的长度分别是p和q。
    \begin{displaymath}
    p=\sqrt{\frac{2}{(A+B)+\sqrt{(A-B)^{2}+4C^{2}}}}
    \end{displaymath}
    \begin{displaymath}
    q=\sqrt{\frac{2}{(A+B)-\sqrt{(A-B)^{2}+4C^{2}}}}
    \end{displaymath}
    
\subsubsection{凸率}
为目标区域面积与目标区域凸包面积之比,该特征包含着描述边界不规则特性的信息。
    \begin{displaymath}
    C_{R}=\frac{A}{\sum_{x=1}^{M}\sum_{y=1}^{N}k(x,y)}
    \end{displaymath}
    分母为凸包区域的面积。
    
\subsubsection{密集度}
描述目标密集度的量化特征,提供了目标形状的重要信息。在周长确定后,密集度越高,所围成的面积越大。
    \begin{displaymath}
    C_{2}=\frac{L^{2}}{4\pi A}
    \end{displaymath}
    L为周长。
    
\subsubsection{球状性}
内切圆的直径与外接圆的直径之比。
    \begin{displaymath}
    S=\frac{r_{i}}{r_{c}}
    \end{displaymath}
    
\subsubsection{伸长度}
周长与目标区域最小外接矩形面积之比。
    \begin{displaymath}
    P=\frac{L}{WH}
    \end{displaymath}

\subsubsection{叶状性}
叶状反映了边界的幅度特征,为区域重心到边界的最短距离与目标区域的最大宽度之比。
    \begin{displaymath}
    B=\frac{R_{1}}{W_{max}}
    \end{displaymath}
    
\subsection{几种典型的特征描述方法}

\subsubsection{边界描述子}
\begin{itemize}
\item 链码
\item 多边形近似
\item 骨架
\item 形状数
\item 统计矩:边界线段的形状可以通过简单的统计矩进行定量的描述,如均值、方差和高阶矩。
\item 傅里叶描述子
\item 曲率尺度空间
\item 形状上下文(KNN)
\end{itemize}

\subsubsection{区域描述子}
\begin{itemize}
\item 拓扑描述:欧拉数
\item 不变矩
\item 角半径变换(Angular RadialTransformation, ART):通过使用一组半径变换系数,描述单个连通区域或者不连通区域,对旋转和噪声具有鲁棒性。
\item 纹理
    \begin{itemize}
    \item 统计方法:灰度共生矩阵
    \item 模型法:马尔科夫随机场
    \item 频谱方法:Gabor滤波、小波变换
    \end{itemize}
\end{itemize}

\begin{comment}
\subsection{经典特征描述方法}

\subsubsection{SIFT特征}

\subsubsection{HOG特征}

\subsubsection{LBP特征}

\subsubsection{Shape Context}

\subsubsection{Fisher Vector}
\end{comment}

\subsection{特征融合}
特征融合分为三个层次:数据级融合、特征级融合和决策级融合。
数据级融合是结合未加工的信息来得到更加丰富的信息。
特征级融合是选择并结合特征来去除多余和无关的特征。
决策级融合是用多个相同或不同的分类器,相同或不同的分类器。

图像融合方法:

像素级:PCA(主成分分析)、HIS变换、Brovery变换、线性加权法、SFIM、IHS变换、高通滤波法、小波变换融合算法。

特征级:聚类分析法、贝叶斯估计法、信息熵法、神经网络法、带权平均法、Dempster-shafer推理法、表决法及神经网络法。

决策级:神经网络法、模糊聚类法、专家系统贝叶斯估计法、模糊集理论、可靠性理论以及逻辑模板法。




\begin{comment}
\begin{itemize}
\item 基于灰度共生矩阵的方法
\item 灰度-梯度共生矩阵分析法
\item 灰度行程长度统计法
\item 小波分析法
\item 基于Gabor小波变换的纹理分析法
\end{itemize}
\end{comment}