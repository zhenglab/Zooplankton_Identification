\section{13类浮游动物的特征}
\begin{description}
    \item[Appendicularia(尾海鞘纲)] 属于脊索动物门,体型像蝌蚪,身体分为躯干和尾两部分。躯干为椭圆形;尾部扁平,比躯干要长。\footnote{\url{https://zh.wikipedia.org/wiki/\%e5\%b0\%be\%e6\%b5\%b7\%e9\%9e\%98\%e7\%ba\%b2}} 大小:小于5mm。
    
    观察采集的图像发现:
    \begin{itemize}
        \item 形状像蝌蚪,分为躯干和尾部。
        \item 躯干较大且灰度较深,并不是呈现规则的椭圆(还有部分突出了的东西还不知道是什么)。
        \item 尾部大致呈现两种形状:一种细长弯曲;另一种较粗(粗细甚至于头部差不多),呈现柳叶状。尾部的灰度相比于躯干较浅,轮廓不太清晰。
    \end{itemize}
    \item[Bubble(气泡)] 非生物。
    
    观察采集的图像发现:
        \begin{itemize}
        \item 圆形。
        \item 气泡四周灰度深,中间灰度很浅,呈亮白色。
        \end{itemize}
    \item[Chaetognatha(毛颚动物门)] 全体分为头、躯干和尾3部分。头部略膨大,躯干与头连接处稍微缢缩为颈部,尾部末端尖细。头部两侧各着生一列4~13根镰刀状、几丁质的颚毛。\footnote{\url{https://zh.wikipedia.org/wiki/\%e6\%af\%9b\%e9\%a2\%9a\%e5\%8a\%a8\%e7\%89\%a9\%e9\%97\%a8}}大小:最大成年成年个体长105mm,该类成年个体的长度一般都大于5mm。
    
    观察采集的图像发现:
        \begin{itemize}
        \item 身体修长,可以明显看出身体分为头、躯干和尾三部分。
        \item 头部小且圆滑,在头与躯干连接的地方略窄。
        \item 躯干较粗,轮廓清晰。
        \item 尾部慢慢变窄,末尾尖。
        \end{itemize}
    \item[{\color{blue}Cladocera Penilia(Penilia avirostris,鸟喙尖头溞)}] 属于节肢动物门,鳃足纲,枝角目,俗称水跳蚤。大小:大约为1mm左右。
    
    观察采集的图像发现:
    \begin{itemize}
    \item 身体短小,有两条长长的触角(但并不是每一幅图像中都可以看到。有时触角是向前的,可以看的很清楚;有时触角是向后的,和身体重合在了一起)。
    \item 该类浮游动物身体中轴线的地方灰度较深(感觉类似人体的脊柱),这个颜色较深的中轴线上还有一条条纹理线连向边缘(就像人体脊柱上连着的骨骼)。由于运动,扫描得到的图像中浮游动物的中轴线并不是都在其身体中间。
    %长圆形,大部分图像可以看到一对较长的触角,并且中轴灰度较深。
    \end{itemize}
    \item[Copepoda(桡脚类)] 属于节肢动物门,颚足纲,桡足类属于其下的一个亚纲。体形像泪珠,有大的触角。分为前体部和后体部,前体部较为宽大,后体部较为短小。\footnote{\url{https://zh.wikipedia.org/wiki/\%e6\%a9\%88\%e8\%85\%b3\%e9\%a1\%9e}}前体部前体部由头和胸部组成,头部有两对触角,胸部有鄂足、五对胸足。后体部无附肢,由3—5节组成。最末的腹节称尾节,末端具1对尾叉,尾叉的末端有5根不等长的刚毛,常呈羽状。\footnote{\url{http://baike.baidu.com/view/665478.htm}}
    
    观察采集的图像发现:
    \begin{itemize}
    \item 该类动物身体呈长椭圆形,尾部长在椭圆形一段(由于尾叉末端有几根不等长的刚毛,因此尾部呈一簇),触角长在椭圆的另一端(一共有两对触角,但最多只能看到一两个,有的图像甚至看不到)。
    \item 从该类动物正上方扫描得到的目标关于其自身的中轴对称。从该类动物侧面扫描得到的目标不对称,其身体一侧长着几对胸足。
    \item {\color{blue}该类中有八十几张图片中有多个目标,应该分在Multiple类中。}
    \end{itemize}
    
    \item[{\color{blue}Decapoda(十足目)}] 属于节肢动物门,软甲纲。分为两类:Lucifer hanseni和Crab larvae。体躯延长呈虾形(腹部发达)或缩短扁圆呈蟹形(腹部化)。\footnote{\url{http://baike.baidu.com/link?url=LWmrgD_DVUcw0upg_zi0LTIJWj6quxa_juRrS3zUt91A-FjPM6VQwYfZ5fFZckzIyEGCaXyS_pikXUGg2JsYMXUX-uFEkmkLqC5lfkxvXvApK3WRBcWQkfbDhMlfTdgrWvh-728gSoUylWZG2UstFK}}
    
    观察采集的图像发现:
    \begin{itemize}
    \item 该类浮游动物形状特征并不是很统一,大致分为两类:虾形和蟹形。
    \item 一些图像中可以看到目标有一条尾巴(像虾的尾巴)。
    \item 一些图像中可以看到目标有一对灰度较深的复眼。
    \end{itemize}
    \item[Doliolida(海樽目)] 属于脊索动物门,樽海鞘纲\footnote{\url{https://zh.wikipedia.org/wiki/\%e6\%a8\%bd\%e6\%b5\%b7\%e9\%9e\%98\%e7\%ba\%b2}}。体型一般呈桶状,体壁最外是被囊层,其内层是外套膜。被囊层下有8~9条肌带环绕着体躯。
    
    观察采集的图像发现:
        \begin{itemize}
        \item 由于该类浮游动物比较透明,因此在图像中灰度较浅,并且其桶状轮廓也不完整了,但最明显的是能看到大概7、8条环状的肌肉带,有的图像中还能看到内部器官。
        \end{itemize}
    \item[{\color{blue}Egg}] 鱼卵以及其他浮游动物的卵。
    
    观察采集的图像发现:
        \begin{itemize}
        \item 形状大致都呈圆形。
        \item 有的卵整体灰度都很深;有的卵中间有一块灰度较深的区域,四周灰度较浅(结构像细胞)。
        \item {\color{blue}由于是不同动物的卵,因此其灰度特征差异较大}。
        \end{itemize}
    \item[Fiber(纤维)] 非生物。
        \begin{itemize}
        \item 弯曲的线状,有的纤维有分叉和交叉。
        \item 该类图像中噪声较多,纤维的边缘也不是很规则。
        \end{itemize}
    \item[Gelatinous(明胶)] 胶质的浮游动物,包括Aglaura(属于刺胞动物门,这一个没有搜的中文名字,但也属于水母类)、Medusa(水母,属于刺胞动物门,水螅纲)、Siphonophora(管水母,属于刺胞动物门,水螅纲,管水母目)、Radiolaria(放射虫,属于原声动物门,辐足纲)和Salps(樽海鞘,属于脊索动物门Chordata,樽海鞘纲Thaliacea,纽鳃樽目Salpida)。该类是多种呈胶质浮游动物的集合。大部分水母都有三个主要部位:圆伞状或是钟状(寺院里面敲得那种钟)的身体﹐触器和口腕。
    
    观察采集的图像发现:
        \begin{itemize}
        \item 由于该类呈胶状,因此该类物体灰度整体较浅,边缘也不是十分清晰。大部分是水母,有小部分的樽海鞘(与海樽目形态很相似),小部分的放射虫。
        \item 其中水母也包括很多类,形态大致呈现以下几种:
            \begin{itemize}
            \item 一些水母身体呈现类似钟状(这里呈现钟状有长有短,有粗有细,还有的会发生一点弯曲),灰度较浅,内部有一块颜色较深的椭圆形区域。
            \item 一些水母也呈钟状,但内部没有颜色较深的椭圆形区域,整个身体灰度均匀。
            \item 还有的个头稍微偏小,形状有的类似圆形、像半个胶囊(应该是由于拍摄原因,有的拍到顶部,有的拍到侧面),体内有颜色较深的一个大点和几个小点。(可能是灯塔水母)
            \item 还有四张看不出形状的,不知道是什么。
            \end{itemize}
        \item 放射虫:形状近似圆形(但由于整体灰度较浅,形状保存是完整),中间有一块灰度较深的区域,四周灰度较浅,可以看到淡淡的细纹从中心连接到边界。
        \end{itemize}
    \item[{\color{blue}Multiple(多个生物)}] 由于浮游动物的重叠,导致分割过程中多个浮游动物被分割到一张图像上。
    \item[{\color{blue}Nonbio}] 非生物的集合。(不符合以上集中浮游动物的形态特征)
    \item[Pteropoda(翼足目)] 属于软体动物门,腹足纲。
    
    观察采集的图像发现:
    \begin{itemize}
    \item 该类浮游动物灰度较深,形状总体都呈现一头宽一头窄。
    \item 形状总体呈现三类:有的呈现象牙状,有点弯曲;有的较粗短,像一顶尖的小帽子;有的呈现细长的三角形状。
    \end{itemize}
\end{description}

参考网站:\url{http://www.imas.utas.edu.au/zooplankton/home}。
